\documentclass[11pt]{article}
\usepackage{neurostyle, amssymb,graphicx,amsmath, caption, subcaption}

\graphicspath{ {./images/} }

\begin{document}
\maketitle
\section*{Chosen Algorithm}
One may at times reflect on the peculiar fact that, while we humans understand some of the furthest reaches of the universe,
or the deepest depths of our own oceans, or heck, even category theory, we know very little about what's actually going on in our own heads.
The inner workings of the human brain are an extraordinary frontier. And fascinatingly, the problem is just as much computational
 as it is biological. Our group too has been motivated by this problem for years. Roughly a week ago, our fascination with this problem
 was re-awakened when Dr. Hess gave a talk at our very own Montana State University math seminar. We took a look at one of her quintessential papers,
 entitled ``A Topological Representation of Branching Neuronal Morphologies". Then, we were off and running. 
 
 The main insight behind the paper is that neurons have a tree-like branching structure, and it may be helpful to categorize them based on their branching.
 Namely, it may be pertinent to use techniques from persistent homology to provide some kind of insight into how to categorize neurons. This is what our group
 plans to focus on for our final project in csci 432. Specifically, the Topological Morphology Descriptor (TMD) algorithm outlined in her paper. On a high level, this algorithm
 takes in a neuron embedded in a metric space as input, and computes a radial filtration on the neuron where birth and death events are measured by the persistence of each
 various branch in the neuron. The algorithm ensures as output a persistence barcode, topologically summarizing the tree structure of the neuron. 
 
 In practice, Dr. Hess provides in her talks and in the paper that this simple method of persistent homology is actually often quite effective in the categorization of a neuron.
 As a result, she presents persistent homology and computational topology solutions as important for the future of understanding the structure of intricate pieces of the brain.
 This core algorithm is central for much of her work, and indeed is showing great promise the more it has been investigated. One of the major strengths of her algorithm is its runtime-
 linear in the number of simplices. One possible shortfall however is its space complexity, which, though not named in the algorithm, is dependent on the number of active child nodes down the tree
 (aka neuron) at that moment in the filtration. These nodes correspondingly are saved at each current parent node in the filtration. We believe that the functionality of this filtration may be
 achieved equivalently with a discrete Morse function, which is included in our +1 portion of the project. In fact, a discrete Morse function may not only achieve equivalent performance to the filtration,
 but it may also provide some exciting insights when comparing different types of neurons in terms of their structure in an attempt to shed some light on the reasoning behind different neuron functionalities as
 a whole.

\section*{Proposed Extensions}
As mentioned above, we plan to investigate discrete Morse functions as a potential alternative to the filtration provided in Dr. Hess' TMD algorithm. In doing so, we hope to provide a formal space complexity analysis
of her algorithm, and compare it to known, prevalent algorithms in discrete Morse theory, such as those introduced by Holmgren, et al. We conjecture that using a discrete Morse function in place of her filtration will have
no disadvantages in terms of performance in categorizing neurons, in runtime, and it may in fact produce slight space complexity improvements. This conjecture certainly is to be further investigated, but the authors
are confident in this assertion. So why then would we bother to use a discrete Morse function in place of the already widely accepted filtration? This boils down to an embedded property inherent to a discrete Morse 
function. Morse functions provide an inherent set of instructions as to how to collapse simplicial collapses without altering their topology. We are choosing to utilize discrete Morse functions in this context because they 
may provide insight into differing functions between neurons as a result of their structure. Designating critical points within each specific kind of neuron is revealing when comparing neurons because it allows for simplicial 
collapsing, and as a result, it allows us
to pinpoint specific neighborhoods within neurons which might have implications for different functionality. Little is known about the structure of neurons as they relate to functionality, and if we strategically were to collapse differing
sections of neurons, we could pinpoint differences in different kinds of neurons, and thus more effectively categorize neurons by their structure.
In doing so, we may gain an important insight into the actual local structure of different kinds of neurons, and thus, a better
idea of what structurally differentiates function within different kinds of neurons. We are excited by the possibilities that this project may have in both improving the actual categorization of neurons in terms of space complexity,
as well as structurally pinpointing the active sections within neurons which may potentially determine functionality in specific cases. 

\end{document}
